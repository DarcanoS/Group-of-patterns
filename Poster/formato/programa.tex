\usepackage[]{textpos}
\usepackage{multido}
\usepackage{ifthen}
\usepackage{xargs}
\usepackage{xparse}
\usepackage{color}
\usepackage{lipsum}  
\usepackage{caption}
\usepackage{lmodern}
\usepackage{mathtools,amsfonts}
\usepackage{enumitem}

\usepackage{xcolor}
\usepackage[spanish]{babel}
\usepackage{lipsum}
\usepackage{pgffor}

\usepackage[tikz]{bclogo}
\usepackage[most]{tcolorbox}
\usepackage{varwidth}
\usepackage{ifthen}
\usepackage{xparse}
\usepackage{pgfkeys}
\usetikzlibrary{calc}
\usepackage[explicit]{titlesec}
\usepackage{titletoc}
\usepackage{anyfontsize}
\usepackage{wrapfig}
\usepackage{minted}
\usepackage[inkscapearea=page]{svg}
\setsvg{inkscapeexe={"D:/software/programas/inkscape/bin/inkscape.exe"}}
\usepackage{xfakebold}

\RequirePackage{tcolorbox}
\usetikzlibrary{patterns}

\colorlet{CLinea}{gray!70!white}
\colorlet{CFondo}{gray!5!white}
\colorlet{CBlanco}{white}
\colorlet{CNegro}{black}

\colorlet{CFNombre}{red!80!white}
\colorlet{CBNombre}{red!10!white}

\colorlet{CTituloDeParrafo}{blue!40}
\colorlet{CSubrayadoDeTitulo}{blue!40}

\newcommand{\ConfigurarPaleta}{
%------------COMUNIDAD-----------------
\ifthenelse{\equal{\MODO}{ModoC}}
    {
    \colorlet{CFNombre}{green!80!white}
    \colorlet{CBNombre}{green!10!white}
    \colorlet{CTSeccion}{blue!40}
    \colorlet{CCSeccion}{gray!20}
    \definecolor{CTDiario}{RGB}{133, 146, 158}
    \definecolor{CCDiario}{RGB}{253, 254, 254}
    }{}
%-------------ESTUDIANTES-------------
\ifthenelse{\equal{\MODO}{ModoE}}
    {
    \colorlet{CFNombre}{yellow!80!white}
    \colorlet{CBNombre}{yellow!10!white}    
    \colorlet{CTSeccion}{blue!30}
    \definecolor{CCSeccion}{RGB}{255, 220, 110}   
    \definecolor{CTDiario}{RGB}{133, 193, 233}
    \definecolor{CCDiario}{RGB}{253, 242, 233}
    }{}
}

\usepackage{fontspec}
\usepackage[outline]{contour}
%------------------------CONFIGURARA FUENTES---------------------------
\newcommand{\FuenteTSG}{\fontsize{50}{50}}
\newcommand{\FuenteTG}{\fontsize{40}{40}}  
\newcommand{\FuenteTM}{\large}
\newcommand{\FuenteTP}{\normalsize} 

\newcommand{\FuenteComic}{ka Blam}
\newcommand{\FuenteEstudiante}{Just a Moment}
\newcommand{\FuenteDocente}{calibri}

\newcommand{\FUENTE}{\setmainfont{\FuenteEstudiante}}

\newcommand{\ConfigurarFuentes}{
  \ifthenelse{\equal{\MODO}{ModoE}}
             {%              
              \renewcommand{\FUENTE}{\setmainfont{\FuenteEstudiante}}           
             }{}
    \ifthenelse{\equal{\MODO}{ModoC}}
             {%            
              \renewcommand{\FUENTE}{\sffamily}            
             }{}    
}


\newcommand{\Escudo}{Escudo_UD}

\newcommand{\ConfigurarImagenes}{	
	%-------------ESTUDIANTES-------------
	\ifthenelse{\equal{\MODO}{ModoE}}
	{
	}{}
}


%%%%%%%%%%%%%%%%%%%%%%%%%%%%%%%%%%%%%%%%%%%%%%%%%%%%%%%%%%%%%%%%%%%%
%%%%%%%%%%%%%%%%%%%%%%%%%%%%% DISENO %%%%%%%%%%%%%%%%%%%%%%%%%%%%%%%
%%%%%%%%%%%%%%%%%%%%%%%%%%%%%%%%%%%%%%%%%%%%%%%%%%%%%%%%%%%%%%%%%%%%
\newcommand{\Comillas}[1]{``#1''}

\newcommand{\Negrita}[1]{
	 \ifthenelse{\equal{\MODO}{ModoE}}
	{% 
	\unskip\setBold[0.2]\aftergroup\unsetBold\aftergroup\ignorespaces  #1       
	}{}
	\ifthenelse{\equal{\MODO}{ModoC}}
	{%            
	 \textbf{#1}         
	}{}	
}

\newcommand{\COLOR}{\color{white}}
\newcommand{\MODO}{ModoD}
\newcommand{\MATERIA}{ARQUITECTURA EMPRESARIAL}
\newcommand{\Ciudad}{Bogotá}
\newcommand{\Volumen}{\date}

\newcommand{\Diario}[1]{	
	\FuenteTSG\color{CTDiario}\textbf{#1}	 
}

\titlecontents{section}[2.4pc]
{\addvspace{1pt}}
{\contentslabel[\thecontentslabel]{2.4pc}}
{}
{\dotfill\small \thecontentspage}
[]

\titleformat{\section}{\centering\bf\FuenteTG\COLOR}{\FuenteTG\bf\FUENTE\Alph{section}.#1}{0.5em}{}
%\newcommand{\titleformat{\subsection}{\centering\Large\selectfont\color{CTSeccion}}{\thesection}{0.5em}{}}
\newcommand{\subseccionC}[1]{
{\centering\FUENTE\Large\color{CTSeccion} \textbf{#1} \\ \vspace*{10pt} }
}
\newcommand{\subseccionI}[1]{
{\LARGE\selectfont\color{CTSeccion} #1 \\ \vspace*{10pt} }
}
%-----------------------DIBUJAR LINEA-------------------------------

\NewDocumentCommand{\DibujarLinea}{m m m m m}{
  \setlength{\parindent}{0cm}
  \setlength{\parskip}{0cm}
  %\noindent
  \count3=#1%
  \advance\count3 by 1%
   \multido{\i=0+1}{\count3}{       
        \ifthenelse{\equal{#5}{horizontal}}{
            %------horizontal
		  \begin{textblock}{#2}(0,\i)
			   \rule[0pt]{#3}{#4}
		  \end{textblock}
         }{
            %------vertical
            \begin{textblock}{#2}(\i,0)
			   \rule[0pt]{#3}{#4}
		    \end{textblock}
         }
	}
}

%------------------VARAIBLES GLOBALES-------------------------------

\newdimen\CeldaW
\newdimen\CeldaH
\newcount\nFilas
\newcount\nColus

%-----------------------CALCULAR GRILLA-----------------------------

\NewDocumentCommand{\CalcularGrilla}{m m O{\textwidth} O{\textheight} O{show}}
{
    %------variables------    
	\nFilas=#1 %filas
	\nColus=#2 %columnas	
    
    \CeldaW=#3 %ancho
    \CeldaH=#4 %alto
   %----------------------
	%calculoar celda
    \divide\CeldaH by \nFilas
	\divide\CeldaW by \nColus
	
	\setlength{\TPHorizModule}{\CeldaW}
	\setlength{\TPVertModule}{\CeldaH}
	
	\ifthenelse{\equal{#5}{show}}{
        %dibujar filas
		\DibujarLinea{\nFilas}{\nColus}{\textwidth}{.1pt}{horizontal}
        %dibujar columnas
        \DibujarLinea{\nColus}{\nFilas}{.1pt}{\textheight}{vertical}
	}{}	
}

%-----------------------NUEVA PAGINA-------------------------------

\NewDocumentEnvironment{NuevaPagina}{m m m}
{
 \thispagestyle{empty}
 \clearpage
 \CalcularGrilla{#1}{#2}[\textwidth][\textheight][#3]
}
%contenido
{
 \newpage
 }

 %-----------------------NUEVA PARRAFO-------------------------------
 %-----vvariables--------
 \newdimen\Margen
 \setlength{\Margen}{2pt}
 \newdimen\AltoP
 %------------------------
 \NewDocumentEnvironment{NuevoParrafo}{m m O{1} O{1}}%
 {%
 \setlength{\AltoP}{\dimexpr#4\TPVertModule-2\Margen\relax}%
 \TPMargin{\Margen}%
 \begin{textblock}{#3}(#2,#1)%
 }%
 {%  
 \end{textblock}%
 }%
%-------------------------------------------------------------------------------------------------------------%
%----------------------------------------CAJAS----------------------------------------------------------------%
%-------------------------------------------------------------------------------------------------------------%

\newcommand{\Imagen}[2]{   
    \ifthenelse{\equal{\MODO}{ModoE}}
    {%           
      \includesvg[#1]{formato/apariencia/imagenes/#2Cartoon}
    }{}
    \ifthenelse{\equal{\MODO}{ModoC}}
    {%            
      \includesvg[#1]{formato/apariencia/imagenes/#2}
    }{}   
}

\newcommand{\LineaIzqC}{0}
\newcommand{\LineaIzqU}{1}

\newcommand{\LineaDerC}{0}
\newcommand{\LineaDerU}{1}
\newcommand{\LineaDerD}{2}

\newcommand{\LineaSupC}{0}
\newcommand{\LineaSupU}{1}

\newcommand{\LineaInfC}{0}
\newcommand{\LineaInfU}{1}
\newcommand{\LineaInfD}{2}

\newcommand{\Vacio}{vacio}
\newcommand{\ADer}{right}
\newcommand{\AIzq}{left}
\newcommand{\Trazo}{0.3pt}
\newcommand{\Recta}{0.0pt}
\newcommand{\EspacioADer}{35pt}

\newcommand{\ConfigurarLineas}{
\ifthenelse{\equal{\MODO}{ModoE}}
           {\tikzset{decoration={random steps,segment length=5mm,amplitude=\Trazo}}}
           {\tikzset{decoration={random steps,segment length=5mm,amplitude=\Recta}}}
}
\tcbset{
	codigoMarco/.style n args={5}{
        empty,   
        height=\AltoP,        
        frame code={\path[fill=#5](frame.south west) rectangle (frame.north east);};
    	\ifthenelse{\equal{#1}{0}}{}{\draw[decorate,color=CLinea,line width =#1 pt](frame.north east) -- (frame.north west);};
    	\ifthenelse{\equal{#2}{0}}{}{\draw[decorate,color=CLinea,line width =#2 pt](frame.south east) -- (frame.south west);};
    	\ifthenelse{\equal{#3}{0}}{}{\draw[decorate,color=CLinea,line width =#3 pt](frame.north west) -- (frame.south west);};
        \ifthenelse{\equal{#4}{0}}{}{\draw[decorate,color=CLinea,line width =#4 pt](frame.north east) -- (frame.south east);};       
	}
}
\tcbset{
	codigoLinea/.style n args={1}{
		empty,   
		height=\AltoP,        
		frame code={\path[fill=white](frame.south west) rectangle (frame.north east);}; 
		{\draw[decorate,color=#1,line width =10pt]([yshift=-12pt]frame.north east) -- ([yshift=-12pt]frame.north west);};
	}   
 }
\newtcolorbox{Linea}[1]{%	
	codigoLinea={#1}
}

\newtcolorbox{MarcoSinTitulo}[5]{%	
    codigoMarco={#1}{#2}{#3}{#4}{#5}
}

\newtcolorbox{MarcoConTitulo}[8]{	
    title=#6,
    coltitle=CTituloDeParrafo,fonttitle=\normalsize,
    attach boxed title to top #7={yshift=-\tcboxedtitleheight,},    
    boxed title style={
    size=minimal, top=5pt,left=2pt,right=\EspacioADer,
    underlay={
              \draw[decorate,CSubrayadoDeTitulo,line width=3pt,] ([xshift=3pt,yshift=-5pt]frame.south west)--([xshift=-\EspacioADer,yshift=-5pt]frame.south east);
              \ifthenelse{\equal{#8}{\Vacio}}{}{\node[anchor=north east,outer sep=2pt] at ([yshift=2pt]frame.north east) {\includesvg[width=24pt]{#8}}};
             };
    },    
    %-------------------------------
    codigoMarco={#1}{#2}{#3}{#4}{#5}
}

\NewDocumentEnvironment{Marco}{O{1} O{1} O{1} O{1} O{CFondo} O{\Vacio} O{\Vacio} O{\Vacio}}%
{%
\ifthenelse{\equal{#8}{\Vacio}}
    {\renewcommand{\EspacioADer}{3pt}}
    {\renewcommand{\EspacioADer}{35pt}}%    
\ifthenelse{\equal{#6}{\Vacio}}
    {\begin{MarcoSinTitulo}{#1}{#2}{#3}{#4}{#5}}
    {\begin{MarcoConTitulo}{#1}{#2}{#3}{#4}{#5}{#6}{#7}{#8}\vspace*{20pt}}
}%
{
\ifthenelse{\equal{#6}{\Vacio}}
    {\end{MarcoSinTitulo}}
    {\end{MarcoConTitulo}}%
}%


%---------TITULO
\newcommand{\CabeceraDeSeccion}[4][30]{
\begin{NuevoParrafo}{0}{0}[#1][2]
	\begin{Marco}[\LineaSupU][\LineaInfD][\LineaIzqU][\LineaDerD][#4]
		\renewcommand{\COLOR}{\color{#3}}	
		\section{#2}  
	\end{Marco}
\end{NuevoParrafo}
}

\newcommand{\COLORSS}{\color{white}}
\titleformat{\subsection}{\centering\bf\LARGE\COLORSS}{\filcenter\LARGE\bf\FUENTE\Alph{subsection}.#1}{0.0em}{}
\titlecontents{subsection}[2.4pc]
{\addvspace{1pt}}
{---}
{}
{\dotfill\small \thecontentspage}
[]

\newcommand{\CabeceraDeSubSeccion}[4]{
	\begin{NuevoParrafo}{#1}{0}[8][1]
		\begin{Marco}[\LineaSupC][\LineaInfC][\LineaIzqC][\LineaDerC][CBlanco]
		    \begin{center}
		    	\vspace{-5pt}
		    	\renewcommand{\COLORSS}{\color{#3}}
		    	\subsection{#2}		    	
		    	%\color{#3}\LARGE#2
		    \end{center}		
		\end{Marco}
	\end{NuevoParrafo}
	\begin{NuevoParrafo}{#1}{7}[23][1]	
		\begin{Linea}{#4}
		\end{Linea}
	\end{NuevoParrafo}
}

\newcommand{\CabeceraDeSubSeccionConAncho}[5]{
	\begin{NuevoParrafo}{#1}{0}[8][1]
		\begin{Marco}[\LineaSupC][\LineaInfC][\LineaIzqC][\LineaDerC][CBlanco]
			\begin{center}
				\vspace{-5pt}
				\renewcommand{\COLORSS}{\color{#3}}
				\subsection{#2}		    	
				%\color{#3}\LARGE#2
			\end{center}		
		\end{Marco}
	\end{NuevoParrafo}
	\begin{NuevoParrafo}{#1}{7}[#5][1]	
		\begin{Linea}{#4}
		\end{Linea}
	\end{NuevoParrafo}
}

\newcommand{\CabeceraDeSubSeccionConAnchoN}[7]{
	\begin{NuevoParrafo}{#1}{#2}[16][2]
		\begin{Marco}[\LineaSupC][\LineaInfC][\LineaIzqC][\LineaDerC][CBlanco]
			\begin{center}
				\vspace{-5pt}
				\renewcommand{\COLORSS}{\color{#6}}
				\subsection{\textbf{#5}}		    	
				%\color{#3}\LARGE#2
			\end{center}		
		\end{Marco}
	\end{NuevoParrafo}
	\begin{NuevoParrafo}{#1}{\the\numexpr #2 + #3 \relax}[#4][1]	
		\begin{Linea}{#7}
		\end{Linea}
	\end{NuevoParrafo}
}

\newcommand{\SeparadorDeSubSeccion}[2]{	
	\begin{NuevoParrafo}{#1}{0}[30][1]	
		\begin{Linea}{#2}
		\end{Linea}
	\end{NuevoParrafo}
}

\newcommand{\CabeceraDePoster}[1]{
	\begin{NuevoParrafo}{0}{0}[8][6]
		\begin{Marco}[\LineaSupC][\LineaInfC][\LineaIzqC][\LineaDerC][CCDiario]
		\Imagen{width=65pt}{\Escudo}
		\end{Marco}
	\end{NuevoParrafo}	
	\begin{NuevoParrafo}{0}{8}[68][6]
		\begin{Marco}[\LineaSupC][\LineaInfC][\LineaIzqC][\LineaDerC][CCDiario]		  
		    \vspace{15pt}
		    \centering\FuenteTSG\color{CTDiario}\textbf{#1}							  
		\end{Marco}
	\end{NuevoParrafo}
}
\newcommand{\CiudadFechaVolumen}{
	\begin{NuevoParrafo}{6}{0}[94][2]
		\begin{Marco}[\LineaSupU][\LineaInfU][\LineaIzqC][\LineaDerC][CBlanco]
		\centering\Negrita{ --- \Ciudad\ \today\ \Volumen \ ---}
		\end{Marco}
	\end{NuevoParrafo}
}

\newcommand{\Pelicula}[3]{	
	\begin{minipage}[t]{9.2cm}
		\color{CTCine}{\Negrita{\large#1}}\begin{center}\includegraphics[width=5cm,height=6cm]{#2}\end{center}#3
	\end{minipage}
}

\newcommand{\Escritor}[3]{	
	\begin{minipage}[t]{18cm}
		\color{CTCine}{\Negrita{\large#1}}\\
		\begin{wrapfigure}{l}{0.3\textwidth}			
			\centering\includegraphics[width=4cm]{#2}			
		\end{wrapfigure}#3		
	\end{minipage}
}

\newcommand{\Personaje}[5]{	
	\begin{minipage}[t]{#1}
		\color{#2}{\Negrita{\large#3}}\\
		\begin{wrapfigure}{l}{4.5cm}			
			\centering\includegraphics[width=4cm]{#4}			
		\end{wrapfigure}#5		
	\end{minipage}
}

\newcommand{\SuperX}[3]{	
	\begin{minipage}[t]{29cm}
		\color{CTCine}{\Negrita{\large#1}}\\
		\begin{wrapfigure}{l}{0.2\textwidth}
			\vspace*{-20pt}			
			\centering\includegraphics[width=4cm]{#2}			
		\end{wrapfigure}#3		
	\end{minipage}
}

\newcommand*{\Obra}[1]{%
	\raisebox{-.3\baselineskip}{%
		\includegraphics[
		height=\baselineskip,
		width=\baselineskip,
		keepaspectratio,
		]{#1}%
	}%
}

\newcommand{\Libro}[2]{
		\begin{minipage}[t]{4.5cm}			
			\centering
			\includegraphics[width=3cm,height=4cm]{#1}\\{\small#2}
		\end{minipage}
	    \vspace{+\parskip}	
}%
\NewDocumentEnvironment{Libros}{m}%
{
\begin{minipage}[t]{11cm}
	\color{CTCine}{\Negrita{\large#1}}\newline\newline	
	\centering\hfill
}%
{
\end{minipage}
}%

%------------------------CONFIURACION DEL MODO-----------------------------------
%\input{formato/progsPython}
\newcommand{\Configuracion}[1]{
    \renewcommand{\MODO}{#1}%para elgir galeria
    \ConfigurarFuentes
    \FUENTE
    \ConfigurarLineas
    \ConfigurarPaleta    
    \ConfigurarImagenes    
}

\newcommand{\ciclolipsum}[1]{
\foreach \x in {1,...,#1}
{
	\lipsum[1][5]
}
\par
}
\newcommand{\cortelipsum}{
\vfill\null	
\columnbreak 
}
\newcommand{\nuevalinea}{\leavevmode\\}

\def\ImagenCircular[#1]#2#3{
	\begin{tikzpicture}
		\clip (0,0) circle (#1);
		\node (0,0) {\includegraphics[width=#2]{#3}};
	\end{tikzpicture}%
}

\newcommand{\autores}[3]{ 
\vspace{-5pt}
\large \centering{UNIVERSIDAD DISTRITAL FJC}
\vspace{5pt}
\hrule
\vspace{5pt}
 #1\\#2 \\#3
}

\newenvironment{MisionVision}[1][Mision]{ 
	\begin{minipage}[t][.20\textheight]{\textwidth}
		\begin{center}
			\color{gray}\bfseries{\fontsize{20}{10}\selectfont #1}
			\vspace{-10pt}
			%\subsubsection{#1}
		\end{center}	
	}
{
	\end{minipage}\\[1cm]
}
\newenvironment{Vision}[1][Vision]{ 
	\begin{minipage}[t][.20\textheight]{\textwidth}
		\begin{center}
			%\color{gray}\bfseries{\fontsize{25}{10}\selectfont #1}
			\section{#1}
		\end{center}	
	}
	{
	\end{minipage}\\[2cm]
}
\newenvironment{Objetivos}[1][Objetivos]{ 
	\begin{minipage}[t][.20\textheight]{\textwidth}
		\begin{center}
			\color{gray}\bfseries{\fontsize{20}{10}\selectfont #1}
			\vspace{-18pt}
			%\color{gray}\bfseries{\fontsize{25}{10}\selectfont #1}
			%\section{#1}
		\end{center}	
	}
	{
	\end{minipage}\\[1cm]
}